\documentclass[a4paper,onecolumn]{article}
\usepackage{hyperref}
% for setting the linespace (\setstretch)
%\usepackage{setspace}
% distance between the columns
%\setlength{\columnsep}{1cm}
% for comments
\usepackage{verbatim}
% filling with lipsum text
\usepackage{lipsum}
% ams
\usepackage{amssymb,amsfonts,amsmath,bm}
\usepackage[pdftex]{graphicx}
\bibliographystyle{plain}

\usepackage[top=2cm, bottom=2cm, left=3cm, right=3cm]{geometry}
\usepackage{fancyhdr}
\pagestyle{fancy}

\usepackage[utf8]{inputenc}
\usepackage[T1]{fontenc}

\title{mcDCA v0.3.1}
\author{
        Simone Marsili \\
        simo.marsili@gmail.com
}
\date{\today}


\begin{document}
\maketitle
%\abstract{\lipsum[78]}

Thanks for donwloading the mcDCA source code.

\section{Requirements}
\label{sec:requirements}

In order to compile mcDCA, you will need to have a Fortran compiler installed on your machine.   
If you are using Debian or a Debian derivative such as Ubuntu, you can install the gfortran compiler using the following command:
\begin{verbatim}
>> sudo apt-get install gfortran
\end{verbatim}

The inference algorithm works by simulating a swarm of persistent Markov chains. 
To compile mcDCA with support for parallel runs on a distributed-memory architecture,
you will need to have a valid MPI implementation installed on your machine. 
The code has been tested and is known to work with the latest versions of both OpenMPI and MPICH.   
OpenMPI (recommended) can be installed on Debian derivatives typing:
\begin{verbatim}
sudo apt-get install openmpi-bin libopenmpi1.10 libopenmpi-dev
\end{verbatim}
(For details on running MPI jobs with OpenMPI see \href{https://www.open-mpi.org/faq/?category=running}{this link}).
\\\

The compiling and linking of source files is handled by Gnu Make. 
If you are using Debian or a Debian derivative such as Ubuntu, you should find 4.1 already installed.

\section{Compiling}
\label{sec:compiling}
From a Unix terminal, type \verb|make| in the \verb|src| directory:
\begin{verbatim}
>> cd src
>> make
\end{verbatim}
After compiling, you should see three binary files: \verb|mcdca-learn|, \verb|mcdca-eval| and \verb|mcdca-learn| in the \verb|src| directory. 
To install the binaries mcDCA in \verb|/usr/local/bin|, type:
\begin{verbatim}
>> make install
\end{verbatim}
and \verb|make uninstall| to uninstall.
The install path can be specified on the make command line as an absolute path,
e.g. :
\begin{verbatim}
>> make install INSTALLDIR=~/.local/bin
\end{verbatim}

\section{mcdca-learn}
\label{sec:mcdca-learn}
\verb|mcdca-learn| implements the \verb|mcdca| routines for statistical inference.
Given a MSA file in FASTA format, \verb|msa.fa|, we can retrieve the parameters of a pairwise model for the sequence distribution. 
\begin{verbatim}
>> mcdca-learn --fasta msa.fa --learn-agd 100 --nsweeps 1000 --lambda 0.01
\end{verbatim}
\begin{itemize}
\item \verb|--fasta msa.fa|: read sequence data from file \verb|msa.fa|
\item \verb|--learn-agd 100|: perform 100 iterations of accelerated gradient descent of the cost function
\item \verb|--nsweeps 1000|: at each iteration, the gradient of the cost function is estimated from a $1000-$sweeps long MCMC trajectory
\item \verb|--lambda 0.01|: the cost function contains a $l_2$ regularization term: higher values of \verb|--lambda| correspond to more strongly regularized solutions
\end{itemize}
\begin{verbatim}
>> mpiexec -n 4 mcdca-learn --nsweeps $NS --fasta msa.fa --learn-agd $NITER --lambda 0.01
\end{verbatim}
From a Unix terminal, type \verb|make| in the \verb|src| directory:

\section{mcdca-eval}
\label{sec:mcdca-eval}
From a Unix terminal, type \verb|make| in the \verb|src| directory:

\section{mcdca-sample}
\label{sec:mcdca-sample}
From a Unix terminal, type \verb|make| in the \verb|src| directory:

\section{System Model}
\label{sec:system-model}

\subsection{Concept}
\lipsum[1]\cite{asimov1951foundation}

\begin{itemize}
	\item \lipsum[53]
	\item \lipsum[11]
\end{itemize}

\lipsum[43]

\subsection{Implementation}
\lipsum[2]

\section{Conclusion}
\label{sec:conclusion}

\lipsum[9-10]

\bibliography{ms}

\end{document}
